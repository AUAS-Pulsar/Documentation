\documentclass[]{article}
\usepackage{lmodern}
\usepackage{amssymb,amsmath}
\usepackage{ifxetex,ifluatex}
\usepackage{fixltx2e} % provides \textsubscript
\ifnum 0\ifxetex 1\fi\ifluatex 1\fi=0 % if pdftex
  \usepackage[T1]{fontenc}
  \usepackage[utf8]{inputenc}
\else % if luatex or xelatex
  \ifxetex
    \usepackage{mathspec}
  \else
    \usepackage{fontspec}
  \fi
  \defaultfontfeatures{Ligatures=TeX,Scale=MatchLowercase}
\fi
% use upquote if available, for straight quotes in verbatim environments
\IfFileExists{upquote.sty}{\usepackage{upquote}}{}
% use microtype if available
\IfFileExists{microtype.sty}{%
\usepackage[]{microtype}
\UseMicrotypeSet[protrusion]{basicmath} % disable protrusion for tt fonts
}{}
\PassOptionsToPackage{hyphens}{url} % url is loaded by hyperref
\usepackage[unicode=true]{hyperref}
\hypersetup{
            pdfborder={0 0 0},
            breaklinks=true}
\urlstyle{same}  % don't use monospace font for urls
\usepackage{color}
\usepackage{fancyvrb}
\newcommand{\VerbBar}{|}
\newcommand{\VERB}{\Verb[commandchars=\\\{\}]}
\DefineVerbatimEnvironment{Highlighting}{Verbatim}{commandchars=\\\{\}}
% Add ',fontsize=\small' for more characters per line
\newenvironment{Shaded}{}{}
\newcommand{\KeywordTok}[1]{\textcolor[rgb]{0.00,0.44,0.13}{\textbf{#1}}}
\newcommand{\DataTypeTok}[1]{\textcolor[rgb]{0.56,0.13,0.00}{#1}}
\newcommand{\DecValTok}[1]{\textcolor[rgb]{0.25,0.63,0.44}{#1}}
\newcommand{\BaseNTok}[1]{\textcolor[rgb]{0.25,0.63,0.44}{#1}}
\newcommand{\FloatTok}[1]{\textcolor[rgb]{0.25,0.63,0.44}{#1}}
\newcommand{\ConstantTok}[1]{\textcolor[rgb]{0.53,0.00,0.00}{#1}}
\newcommand{\CharTok}[1]{\textcolor[rgb]{0.25,0.44,0.63}{#1}}
\newcommand{\SpecialCharTok}[1]{\textcolor[rgb]{0.25,0.44,0.63}{#1}}
\newcommand{\StringTok}[1]{\textcolor[rgb]{0.25,0.44,0.63}{#1}}
\newcommand{\VerbatimStringTok}[1]{\textcolor[rgb]{0.25,0.44,0.63}{#1}}
\newcommand{\SpecialStringTok}[1]{\textcolor[rgb]{0.73,0.40,0.53}{#1}}
\newcommand{\ImportTok}[1]{#1}
\newcommand{\CommentTok}[1]{\textcolor[rgb]{0.38,0.63,0.69}{\textit{#1}}}
\newcommand{\DocumentationTok}[1]{\textcolor[rgb]{0.73,0.13,0.13}{\textit{#1}}}
\newcommand{\AnnotationTok}[1]{\textcolor[rgb]{0.38,0.63,0.69}{\textbf{\textit{#1}}}}
\newcommand{\CommentVarTok}[1]{\textcolor[rgb]{0.38,0.63,0.69}{\textbf{\textit{#1}}}}
\newcommand{\OtherTok}[1]{\textcolor[rgb]{0.00,0.44,0.13}{#1}}
\newcommand{\FunctionTok}[1]{\textcolor[rgb]{0.02,0.16,0.49}{#1}}
\newcommand{\VariableTok}[1]{\textcolor[rgb]{0.10,0.09,0.49}{#1}}
\newcommand{\ControlFlowTok}[1]{\textcolor[rgb]{0.00,0.44,0.13}{\textbf{#1}}}
\newcommand{\OperatorTok}[1]{\textcolor[rgb]{0.40,0.40,0.40}{#1}}
\newcommand{\BuiltInTok}[1]{#1}
\newcommand{\ExtensionTok}[1]{#1}
\newcommand{\PreprocessorTok}[1]{\textcolor[rgb]{0.74,0.48,0.00}{#1}}
\newcommand{\AttributeTok}[1]{\textcolor[rgb]{0.49,0.56,0.16}{#1}}
\newcommand{\RegionMarkerTok}[1]{#1}
\newcommand{\InformationTok}[1]{\textcolor[rgb]{0.38,0.63,0.69}{\textbf{\textit{#1}}}}
\newcommand{\WarningTok}[1]{\textcolor[rgb]{0.38,0.63,0.69}{\textbf{\textit{#1}}}}
\newcommand{\AlertTok}[1]{\textcolor[rgb]{1.00,0.00,0.00}{\textbf{#1}}}
\newcommand{\ErrorTok}[1]{\textcolor[rgb]{1.00,0.00,0.00}{\textbf{#1}}}
\newcommand{\NormalTok}[1]{#1}
\usepackage{longtable,booktabs}
% Fix footnotes in tables (requires footnote package)
\IfFileExists{footnote.sty}{\usepackage{footnote}\makesavenoteenv{long table}}{}
\IfFileExists{parskip.sty}{%
\usepackage{parskip}
}{% else
\setlength{\parindent}{0pt}
\setlength{\parskip}{6pt plus 2pt minus 1pt}
}
\setlength{\emergencystretch}{3em}  % prevent overfull lines
\providecommand{\tightlist}{%
  \setlength{\itemsep}{0pt}\setlength{\parskip}{0pt}}
\setcounter{secnumdepth}{0}
% Redefines (sub)paragraphs to behave more like sections
\ifx\paragraph\undefined\else
\let\oldparagraph\paragraph
\renewcommand{\paragraph}[1]{\oldparagraph{#1}\mbox{}}
\fi
\ifx\subparagraph\undefined\else
\let\oldsubparagraph\subparagraph
\renewcommand{\subparagraph}[1]{\oldsubparagraph{#1}\mbox{}}
\fi

% set default figure placement to htbp
\makeatletter
\def\fps@figure{htbp}
\makeatother


\date{}

\begin{document}

\section{1. Getting started}\label{getting-started}

\subsection{1.1 Example filterbank
files}\label{example-filterbank-files}

\subsubsection{Use one of the following filterbank files as an
example:}\label{use-one-of-the-following-filterbank-files-as-an-example}

\begin{itemize}
\tightlist
\item
  ``https://git.dev.ti-more.net/uploads/-/system/personal\_snippet/2/bc063035797e978034adfb6f2da75e70/pspm8.fil''
\item
  ``https://git.dev.ti-more.net/uploads/-/system/personal\_snippet/2/3da35656df8f722441579847974a03cb/pspm16.fil''
\item
  ``https://git.dev.ti-more.net/uploads/-/system/personal\_snippet/2/e6015ec024ad1f53d4c2f39511620db1/pspm32.fil''
\end{itemize}

\subsection{1.2 Import}\label{import}

\begin{verbatim}
from filterbank.filterbank import *
\end{verbatim}

\section{2. Filterbank Tutorial}\label{filterbank-tutorial}

\subsection{2.1 Create a filterbank
object}\label{create-a-filterbank-object}

\begin{verbatim}
filterbank = Filterbank(<PATH TO FILTERBANK FILE>)
\end{verbatim}

This is an example without parameters, see 2.3 for an example with the
parameters. \#\# 2.2 Read the header from filterbank data

\begin{verbatim}
filterbank.header
\end{verbatim}

Header data contains the following:

\begin{longtable}[]{@{}ll@{}}
\toprule
Variable & Description\tabularnewline
\midrule
\endhead
source\_name & name of filterbank file\tabularnewline
P & period of pulsar in ms\tabularnewline
DM & dispersion measure of pulsar\tabularnewline
machine\_id & id of machine used to receive signal data\tabularnewline
telescope\_id & id of telescope used to receive signal
data\tabularnewline
data\_type & type of file
\texttt{filterbank,\ time\ series}\tabularnewline
fch1 & center frequency of first filterbank channel (MHz)\tabularnewline
foff & filterbank channel bandwidth (MHz)\tabularnewline
nchans & number of filterbank channels\tabularnewline
nbits & number of bits per time sample
\texttt{8,\ 16\ or\ 32}\tabularnewline
tstart & timestamp of first sample (MJD)\tabularnewline
nifs & number of seperate intermediate-frequency channels\tabularnewline
\bottomrule
\end{longtable}

freq\_range = fch1 + (foff * nchans + fch1)\\
time\_range = tstart + (tsamp/24/60/60)

freq\_range is a tuple with a frequency start and a frequency stop\\
The same applies to time\_range

The header data, including the center frequency, can be retrieved by
calling the header attribute from the Filterbank object.

\subsection{2.3 Read filterbank file}\label{read-filterbank-file}

The attributes time\_range and freq\_range can be passed as parameters
to select a specific portion of the filterbank file. To make the
Filterbank object read the filterbank file at once, set the
\texttt{read\_all} parameter to \texttt{True}.

\begin{verbatim}
filterbank = Filterbank(<PATH TO FILTERBANK FILE>, freq_range, time_range, read_all)
\end{verbatim}

\subsection{2.4 Select a range of data from the filterbank
file}\label{select-a-range-of-data-from-the-filterbank-file}

The select\_data method can be used to retrieve data from the Filterbank
object. The user has the option to give a \texttt{time} and/or
\texttt{frequency} range to select a subset from the entire dataset. The
time-range can either be an index of a sample or a float that is
represents a moment in time (in seconds).

\begin{verbatim}
filterbank.select_data(freq_range, time_range)
\end{verbatim}

The \texttt{select\_data} method returns an array of all different
channels/frequencies and a large matrix with all the received radio
signals.

The matrix contains for each time sample an array which has the
intensity per channel/frequency.

\subsection{2.5 Read filterbank as
stream}\label{read-filterbank-as-stream}

When reading the filterbank file as a stream, the user should let the
\texttt{read\_all}-parameter stay \texttt{False} when initializing the
filterbank object.

Each time the user calls the \texttt{next\_row} method, it will retrieve
an array with intensitiy per frequency for a new time sample from the
filterbank file. When the last iteration of the filterbank is reached,
the new\_row method will return \texttt{False}.\\
The same goes for the \texttt{next\_n\_rows} method, where the user is
able to define the amount of rows that should be returned.

\begin{verbatim}
filterbank.next_row()

filterbank.next_n_rows(n_rows=10)
\end{verbatim}

\section{3. Fourier Transformations}\label{fourier-transformations}

\subsection{3.1 DFT}\label{dft}

The \texttt{fourier.dft\_slow} function is a plain implementation of
discrete Fourier transformation.

The \texttt{fourier.fft\_vectorized} function depends on this function.

\subsubsection{3.1.1 Parameters}\label{parameters}

\begin{longtable}[]{@{}ll@{}}
\toprule
Parameter & Description\tabularnewline
\midrule
\endhead
input\_data & Array containing the values to be
transformed.\tabularnewline
\bottomrule
\end{longtable}

Returns an array containing the transformed values.

\subsubsection{3.1.2 Example usage}\label{example-usage}

\begin{Shaded}
\begin{Highlighting}[]
\OperatorTok{>>>} \ImportTok{from}\NormalTok{ Asteria }\ImportTok{import}\NormalTok{ fourier}
\OperatorTok{>>>}\NormalTok{ fourier.dft_slow([}\DecValTok{1}\NormalTok{,}\DecValTok{2}\NormalTok{,}\DecValTok{3}\NormalTok{,}\DecValTok{4}\NormalTok{])}
\NormalTok{array([}\DecValTok{10}\NormalTok{.}\OperatorTok{+}\FloatTok{0.}\NormalTok{00000000e}\OperatorTok{+}\NormalTok{00j, }\OperatorTok{-}\DecValTok{2}\NormalTok{.}\OperatorTok{+}\FloatTok{2.}\NormalTok{00000000e}\OperatorTok{+}\NormalTok{00j, }\OperatorTok{-}\DecValTok{2}\NormalTok{.}\OperatorTok{-}\FloatTok{9.}\NormalTok{79717439e}\OperatorTok{-}\NormalTok{16j, }\OperatorTok{-}\DecValTok{2}\NormalTok{.}\OperatorTok{-}\FloatTok{2.}\NormalTok{00000000e}\OperatorTok{+}\NormalTok{00j])}
\end{Highlighting}
\end{Shaded}

\subsection{3.2 FFT}\label{fft}

The \texttt{fourier.fft\_vectorized} function is a vectorized,
non-recursive version of the Cooley-Tukey FFT

Gives the same result as \texttt{fourier.dft\_slow} but is many times
faster.

\subsubsection{3.2.1 Parameters}\label{parameters-1}

\begin{longtable}[]{@{}ll@{}}
\toprule
Parameter & Description\tabularnewline
\midrule
\endhead
input\_data & Array containing the values to be
transformed.\tabularnewline
\bottomrule
\end{longtable}

Returns an array containing the transformed values.

\subsubsection{3.2.2 Example usage}\label{example-usage-1}

\begin{Shaded}
\begin{Highlighting}[]
\OperatorTok{>>>} \ImportTok{from}\NormalTok{ Asteria }\ImportTok{import}\NormalTok{ fourier}
\OperatorTok{>>>}\NormalTok{ fourier.fft_vectorized([}\DecValTok{1}\NormalTok{,}\DecValTok{2}\NormalTok{,}\DecValTok{3}\NormalTok{,}\DecValTok{4}\NormalTok{])}
\NormalTok{array([}\DecValTok{10}\NormalTok{.}\OperatorTok{+}\FloatTok{0.}\NormalTok{00000000e}\OperatorTok{+}\NormalTok{00j, }\OperatorTok{-}\DecValTok{2}\NormalTok{.}\OperatorTok{+}\FloatTok{2.}\NormalTok{00000000e}\OperatorTok{+}\NormalTok{00j, }\OperatorTok{-}\DecValTok{2}\NormalTok{.}\OperatorTok{-}\FloatTok{9.}\NormalTok{79717439e}\OperatorTok{-}\NormalTok{16j, }\OperatorTok{-}\DecValTok{2}\NormalTok{.}\OperatorTok{-}\FloatTok{2.}\NormalTok{00000000e}\OperatorTok{+}\NormalTok{00j])}
\end{Highlighting}
\end{Shaded}

\section{4. Plotting diagrams}\label{plotting-diagrams}

\subsection{4.1 PSD}\label{psd}

The \texttt{plot.psd} function can be used to generate the data for a
Power Spectral Density plot.

\subsubsection{4.1.1 Parameters}\label{parameters-2}

\begin{longtable}[]{@{}ll@{}}
\toprule
\begin{minipage}[b]{0.15\columnwidth}\raggedright\strut
Parameter\strut
\end{minipage} & \begin{minipage}[b]{0.80\columnwidth}\raggedright\strut
Description\strut
\end{minipage}\tabularnewline
\midrule
\endhead
\begin{minipage}[t]{0.15\columnwidth}\raggedright\strut
samples\strut
\end{minipage} & \begin{minipage}[t]{0.80\columnwidth}\raggedright\strut
1-D array or sequence. Array or sequence containing the data to be
plotted.\strut
\end{minipage}\tabularnewline
\begin{minipage}[t]{0.15\columnwidth}\raggedright\strut
nfft\strut
\end{minipage} & \begin{minipage}[t]{0.80\columnwidth}\raggedright\strut
Integer, optional. The number of bins to be used. Defaults to 256.\strut
\end{minipage}\tabularnewline
\begin{minipage}[t]{0.15\columnwidth}\raggedright\strut
sample\_rate\strut
\end{minipage} & \begin{minipage}[t]{0.80\columnwidth}\raggedright\strut
Integer, optional. The sample rate of the input data in
\texttt{samples}. Defaults to 2.\strut
\end{minipage}\tabularnewline
\begin{minipage}[t]{0.15\columnwidth}\raggedright\strut
window\strut
\end{minipage} & \begin{minipage}[t]{0.80\columnwidth}\raggedright\strut
Callable, optional. The window function to be used. Defaults to
\texttt{plot.window\_hanning}.\strut
\end{minipage}\tabularnewline
\begin{minipage}[t]{0.15\columnwidth}\raggedright\strut
sides\strut
\end{minipage} & \begin{minipage}[t]{0.80\columnwidth}\raggedright\strut
\{`default', `onesided', `twosided'\}. Specifies which sides of the
spectrum to return. Default gives the default behavior, which returns
one-sided for real data and both for complex data. `onesided' forces the
return of a one-sided spectrum, while `twosided' forces two-sided.\strut
\end{minipage}\tabularnewline
\bottomrule
\end{longtable}

\subsubsection{4.1.2 Returns}\label{returns}

\begin{longtable}[]{@{}ll@{}}
\toprule
\begin{minipage}[b]{0.15\columnwidth}\raggedright\strut
Variable\strut
\end{minipage} & \begin{minipage}[b]{0.80\columnwidth}\raggedright\strut
Description\strut
\end{minipage}\tabularnewline
\midrule
\endhead
\begin{minipage}[t]{0.15\columnwidth}\raggedright\strut
Pxx\strut
\end{minipage} & \begin{minipage}[t]{0.80\columnwidth}\raggedright\strut
1-D array. The values for the power spectrum before scaling (real
valued).\strut
\end{minipage}\tabularnewline
\begin{minipage}[t]{0.15\columnwidth}\raggedright\strut
freqs\strut
\end{minipage} & \begin{minipage}[t]{0.80\columnwidth}\raggedright\strut
1-D array. The frequencies corresponding to the elements in
\texttt{Pxx}.\strut
\end{minipage}\tabularnewline
\bottomrule
\end{longtable}

\subsubsection{4.1.3 Example Usage}\label{example-usage-2}

\begin{Shaded}
\begin{Highlighting}[]
\ImportTok{from}\NormalTok{ filterbank.filterbank }\ImportTok{import}\NormalTok{ Filterbank}
\ImportTok{import}\NormalTok{ matplotlib.pyplot }\ImportTok{as}\NormalTok{ plt}
\ImportTok{import}\NormalTok{ numpy }\ImportTok{as}\NormalTok{ np}
\ImportTok{from}\NormalTok{ plot }\ImportTok{import}\NormalTok{ psd}
\ImportTok{from}\NormalTok{ filterbank.header }\ImportTok{import}\NormalTok{ read_header}

\CommentTok{# Instatiate the filterbank reader and point to the filterbank file}
\NormalTok{fb }\OperatorTok{=}\NormalTok{ Filterbank(filename}\OperatorTok{=}\StringTok{'examples/pspm32.fil'}\NormalTok{)}

\CommentTok{# read the data in the filterbank file}
\NormalTok{_, samples }\OperatorTok{=}\NormalTok{ fb.select_data()}

\CommentTok{# Convert 2D Array to 1D Array with complex numbers}
\NormalTok{samples }\OperatorTok{=}\NormalTok{ samples[}\DecValTok{0}\NormalTok{] }\OperatorTok{+}\NormalTok{ (samples[}\DecValTok{1}\NormalTok{] }\OperatorTok{*}\NormalTok{ 1j)}

\CommentTok{# Read the header of the filterbank file}
\NormalTok{header }\OperatorTok{=}\NormalTok{ read_header(}\StringTok{'examples/pspm32.fil'}\NormalTok{)}

\CommentTok{# Calculate the center frequency with the data in the header}
\NormalTok{center_freq }\OperatorTok{=}\NormalTok{ header[b}\StringTok{'fch1'}\NormalTok{] }\OperatorTok{+} \BuiltInTok{float}\NormalTok{(header[b}\StringTok{'nchans'}\NormalTok{]) }\OperatorTok{*}\NormalTok{ header[b}\StringTok{'foff'}\NormalTok{] }\OperatorTok{/} \FloatTok{2.0}

\BuiltInTok{print}\NormalTok{(center_freq)}
\CommentTok{# Get the powerlevels and the frequencies}
\NormalTok{PXX, freqs, _ }\OperatorTok{=}\NormalTok{ psd(samples, nfft}\OperatorTok{=}\DecValTok{1024}\NormalTok{, sample_rate}\OperatorTok{=}\DecValTok{80}\NormalTok{, sides}\OperatorTok{=}\StringTok{'twosided'}\NormalTok{)}

\CommentTok{# Calculate the powerlevel dB's}
\NormalTok{power_levels }\OperatorTok{=}  \DecValTok{10} \OperatorTok{*}\NormalTok{ np.log10(PXX}\OperatorTok{/}\NormalTok{(}\DecValTok{80}\NormalTok{))}

\CommentTok{# Add the center frequency to the frequencies so they match the actual frequencies}
\NormalTok{freqs }\OperatorTok{=}\NormalTok{ freqs }\OperatorTok{+}\NormalTok{ center_freq}

\CommentTok{# Plot the PSD}
\NormalTok{plt.grid(}\VariableTok{True}\NormalTok{)}
\NormalTok{plt.xlabel(}\StringTok{'Frequency (MHz)'}\NormalTok{)}
\NormalTok{plt.ylabel(}\StringTok{'Intensity (dB)'}\NormalTok{)}
\NormalTok{plt.plot(freqs, power_levels)}
\NormalTok{plt.show()}
\end{Highlighting}
\end{Shaded}

\subsection{4.2 Waterfall}\label{waterfall}

The \texttt{plot.waterfall.Waterfall} class can be used to generate
waterfall plots.

\subsubsection{4.2.1 Construction}\label{construction}

\begin{longtable}[]{@{}ll@{}}
\toprule
\begin{minipage}[b]{0.15\columnwidth}\raggedright\strut
Parameter\strut
\end{minipage} & \begin{minipage}[b]{0.80\columnwidth}\raggedright\strut
Description\strut
\end{minipage}\tabularnewline
\midrule
\endhead
\begin{minipage}[t]{0.15\columnwidth}\raggedright\strut
filter\_bank\strut
\end{minipage} & \begin{minipage}[t]{0.80\columnwidth}\raggedright\strut
A \texttt{filterbank} object.\strut
\end{minipage}\tabularnewline
\begin{minipage}[t]{0.15\columnwidth}\raggedright\strut
center\_freq\strut
\end{minipage} & \begin{minipage}[t]{0.80\columnwidth}\raggedright\strut
The center frequency of the signal in the filterbank object\strut
\end{minipage}\tabularnewline
\begin{minipage}[t]{0.15\columnwidth}\raggedright\strut
sample\_freq\strut
\end{minipage} & \begin{minipage}[t]{0.80\columnwidth}\raggedright\strut
The sample frequency of the signal in the filterbank object\strut
\end{minipage}\tabularnewline
\begin{minipage}[t]{0.15\columnwidth}\raggedright\strut
fig\strut
\end{minipage} & \begin{minipage}[t]{0.80\columnwidth}\raggedright\strut
An imaging object, like \texttt{pyplot.figure()}\strut
\end{minipage}\tabularnewline
\begin{minipage}[t]{0.15\columnwidth}\raggedright\strut
mode\strut
\end{minipage} & \begin{minipage}[t]{0.80\columnwidth}\raggedright\strut
String \texttt{\{discrete,\ stream\}}. The mode to operate on. Use
discrete for discrete datasets, and stream for stream data. Defaults to
\texttt{stream}.\strut
\end{minipage}\tabularnewline
\bottomrule
\end{longtable}

\subsubsection{4.2.2 Methods}\label{methods}

\begin{longtable}[]{@{}ll@{}}
\toprule
\begin{minipage}[b]{0.27\columnwidth}\raggedright\strut
Method\strut
\end{minipage} & \begin{minipage}[b]{0.67\columnwidth}\raggedright\strut
Description\strut
\end{minipage}\tabularnewline
\midrule
\endhead
\begin{minipage}[t]{0.27\columnwidth}\raggedright\strut
init\_plot(self)\strut
\end{minipage} & \begin{minipage}[t]{0.67\columnwidth}\raggedright\strut
Initialize the plot\strut
\end{minipage}\tabularnewline
\begin{minipage}[t]{0.27\columnwidth}\raggedright\strut
update\_plot\_labes(self)\strut
\end{minipage} & \begin{minipage}[t]{0.67\columnwidth}\raggedright\strut
Generate the plot labels\strut
\end{minipage}\tabularnewline
\begin{minipage}[t]{0.27\columnwidth}\raggedright\strut
get\_next(self)\strut
\end{minipage} & \begin{minipage}[t]{0.67\columnwidth}\raggedright\strut
Returns the next row of data in the filterbank object\strut
\end{minipage}\tabularnewline
\begin{minipage}[t]{0.27\columnwidth}\raggedright\strut
get\_image(self)\strut
\end{minipage} & \begin{minipage}[t]{0.67\columnwidth}\raggedright\strut
Returns the image data of the full dataset, if using a discrete
dataset.\strut
\end{minipage}\tabularnewline
\begin{minipage}[t]{0.27\columnwidth}\raggedright\strut
update(self, i)\strut
\end{minipage} & \begin{minipage}[t]{0.67\columnwidth}\raggedright\strut
Updates the image with the next row of data, when using a continuous
datastream.\strut
\end{minipage}\tabularnewline
\begin{minipage}[t]{0.27\columnwidth}\raggedright\strut
animated\_plotter(self)\strut
\end{minipage} & \begin{minipage}[t]{0.67\columnwidth}\raggedright\strut
Returns the figure and update function for matplotlib animation\strut
\end{minipage}\tabularnewline
\begin{minipage}[t]{0.27\columnwidth}\raggedright\strut
get\_center\_freq(self)\strut
\end{minipage} & \begin{minipage}[t]{0.67\columnwidth}\raggedright\strut
Returns the center frequency stored in the filterbank header\strut
\end{minipage}\tabularnewline
\bottomrule
\end{longtable}

\subsubsection{4.2.3 Example Usage}\label{example-usage-3}

\paragraph{4.2.3.1 With discrete data}\label{with-discrete-data}

\begin{Shaded}
\begin{Highlighting}[]
\ImportTok{import}\NormalTok{ matplotlib.animation }\ImportTok{as}\NormalTok{ animation}
\ImportTok{from}\NormalTok{ filterbank.header }\ImportTok{import}\NormalTok{ read_header}
\ImportTok{from}\NormalTok{ filterbank.filterbank }\ImportTok{import}\NormalTok{ Filterbank}
\ImportTok{from}\NormalTok{ plot }\ImportTok{import}\NormalTok{ waterfall}
\ImportTok{import}\NormalTok{ pylab }\ImportTok{as}\NormalTok{ pyl}
\ImportTok{from}\NormalTok{ plot.plot }\ImportTok{import}\NormalTok{ next_power_of_2}

\NormalTok{fb }\OperatorTok{=}\NormalTok{ Filterbank(filename}\OperatorTok{=}\StringTok{'./pspm32.fil'}\NormalTok{, read_all}\OperatorTok{=}\VariableTok{True}\NormalTok{)}

\NormalTok{wf }\OperatorTok{=}\NormalTok{ waterfall.Waterfall(filter_bank}\OperatorTok{=}\NormalTok{fb, fig}\OperatorTok{=}\NormalTok{pyl.figure(), mode}\OperatorTok{=}\StringTok{"discrete"}\NormalTok{)}

\NormalTok{img }\OperatorTok{=}\NormalTok{ wf.get_image()}

\NormalTok{pyl.show(img)}
\end{Highlighting}
\end{Shaded}

\paragraph{4.2.3.2 With stream data}\label{with-stream-data}

\begin{Shaded}
\begin{Highlighting}[]
\ImportTok{import}\NormalTok{ matplotlib.animation }\ImportTok{as}\NormalTok{ animation}
\ImportTok{from}\NormalTok{ filterbank.header }\ImportTok{import}\NormalTok{ read_header}
\ImportTok{from}\NormalTok{ filterbank.filterbank }\ImportTok{import}\NormalTok{ Filterbank}
\ImportTok{from}\NormalTok{ plot }\ImportTok{import}\NormalTok{ waterfall}
\ImportTok{import}\NormalTok{ pylab }\ImportTok{as}\NormalTok{ pyl}
\ImportTok{from}\NormalTok{ plot.plot }\ImportTok{import}\NormalTok{ next_power_of_2}


\NormalTok{fb }\OperatorTok{=}\NormalTok{ Filterbank(filename}\OperatorTok{=}\StringTok{'./pspm32.fil'}\NormalTok{)}

\NormalTok{wf }\OperatorTok{=}\NormalTok{ waterfall.Waterfall(fb}\OperatorTok{=}\NormalTok{fb, fig}\OperatorTok{=}\NormalTok{pyl.figure(), mode}\OperatorTok{=}\StringTok{"stream"}\NormalTok{)}

\NormalTok{fig, update, frames, repeat }\OperatorTok{=}\NormalTok{ wf.animated_plotter()}

\NormalTok{ani }\OperatorTok{=}\NormalTok{ animation.FuncAnimation(fig, update, frames}\OperatorTok{=}\NormalTok{frames,repeat}\OperatorTok{=}\NormalTok{repeat)}
\NormalTok{pyl.show()}
\end{Highlighting}
\end{Shaded}

\section{5. Timeseries module}\label{timeseries-module}

Learn how to use the timeseries object. The timeseries object is used
for all timeseries related operations both filterbank and non-filterbank
(general array) types.

\subsection{5.1 Initialize the timeseries
object}\label{initialize-the-timeseries-object}

The \texttt{Timeseries} module can be used with both the filterbank file
and a general \texttt{numpy} array.

\subsubsection{5.1.1 Initialize using a filterbank
file}\label{initialize-using-a-filterbank-file}

\begin{itemize}
\tightlist
\item
  Create an filterbank object using the standard filterbank module.
\item
  Read the filterbank file in memory (do not use streams)
\item
  Initialize the timeseries object using the filterbank object.
\end{itemize}

\begin{Shaded}
\begin{Highlighting}[]
\ImportTok{import}\NormalTok{ filterbank.filterbank }\ImportTok{as}\NormalTok{ Filterbank}
\ImportTok{import}\NormalTok{ timeseries.timeseries }\ImportTok{as}\NormalTok{ Timeseries}

\CommentTok{# Read the filterbank file from a file. }
\NormalTok{filterbank_obj }\OperatorTok{=}\NormalTok{ Filterbank(}\StringTok{'./pspm32.fil'}\NormalTok{)}

\CommentTok{# Read the filterbank as a whole instead of as a stream. }
\NormalTok{filterbank_obj }\OperatorTok{=}\NormalTok{  filterbank_obj.read_filterbank()}

\CommentTok{# Initialize the timeseries object. }
\NormalTok{ts }\OperatorTok{=}\NormalTok{ Timeseries().from_filterbank(filterbank_obj)}
\end{Highlighting}
\end{Shaded}

\subsubsection{5.1.2 Initialize using a numpy
array}\label{initialize-using-a-numpy-array}

\begin{Shaded}
\begin{Highlighting}[]
\ImportTok{import}\NormalTok{ numpy }\ImportTok{as}\NormalTok{ np}
\ImportTok{import}\NormalTok{ timeseries.timeseries }\ImportTok{as}\NormalTok{ Timeseries}

\NormalTok{input_array }\OperatorTok{=}\NormalTok{ np.array([}\DecValTok{1}\NormalTok{, }\DecValTok{2}\NormalTok{, }\DecValTok{3}\NormalTok{, }\DecValTok{4}\NormalTok{, }\DecValTok{5}\NormalTok{, }\DecValTok{7}\NormalTok{, }\DecValTok{9}\NormalTok{, }\DecValTok{10}\NormalTok{, }\DecValTok{11}\NormalTok{])}

\NormalTok{ts }\OperatorTok{=}\NormalTok{ Timeseries(input_array)}
\end{Highlighting}
\end{Shaded}

\subsection{5.2 Retrieve the timeseries array from timeseries
object}\label{retrieve-the-timeseries-array-from-timeseries-object}

\subsubsection{5.2.1 Retrieve timeseries
object}\label{retrieve-timeseries-object}

\begin{Shaded}
\begin{Highlighting}[]
\CommentTok{# Assumed that your timeseries object has been initialized. }

\NormalTok{timeseries_array }\OperatorTok{=}\NormalTok{ timeseries.get()}
\end{Highlighting}
\end{Shaded}

\subsection{5.3 Downsample the
timeseries}\label{downsample-the-timeseries}

The first implemented feature for the timeseries object is the
downsample/decimate function. This enables you to downsample your
timeseries by \texttt{q} scale. This will make your input array smaller
and basically `cuts' the other parts off.

\subsubsection{5.3.1 Downsample/Decimate}\label{downsampledecimate}

Called downsample in the current release because no anti-alliasing is
used, might be renamed to decimate once more advanced operations (such
as antialiasing) are used.

\begin{Shaded}
\begin{Highlighting}[]
\CommentTok{# Downsampled array shall be 3 times smaller than the current timeseries (as initialized) }
\NormalTok{scale }\OperatorTok{=} \DecValTok{3}
\CommentTok{# Returns an array with the downsampled timeseries, can also be retreived lated user timseries.get()}
\NormalTok{downsampled_array }\OperatorTok{=}\NormalTok{ timeseries.downsample(scale)}
\end{Highlighting}
\end{Shaded}

\section{6. Clipping}\label{clipping}

\subsection{6.1 Clipping}\label{clipping-1}

The \texttt{clipping.clipping} function can be used to remove noise from
the data.

It combines all the individual methods of the clipping module to do
this. The individual methods are described below in order.

\begin{verbatim}
clipping(<FREQUENCY_CHANNELS>, <TIME_SAMPLES>)
\end{verbatim}

\subsection{6.2 Filter samples}\label{filter-samples}

The \texttt{clipping.filter\_samples} function can be used to remove
entire time samples, that have noise in them.

The noise for a time sample is calculated by summing all the individual
frequencies for one sample, and then comparing it with the average(mean)
of all samples. If the intensity of a sample is lower than
\texttt{average\ intensity\ per\ sample\ *\ factor}, the sample will be
added back to the array. Otherwise, it's removed.

It is recommended to give only the first 2000 samples to this method,
because adding more will only hurt the performance.

\begin{verbatim}
filter_samples(<TIME_SAMPLES>)
\end{verbatim}

\subsection{6.3 Filter channels}\label{filter-channels}

The \texttt{clipping.filter\_channels} function can be used to remove
entire channels/frequencies with noise from the data.

The noise for a frequency channel is identified by calculating the mean
and standard deviation for each column in the data. After that, samples
are removed if the mean or standard deviation of the intensity is higher
than the \texttt{average\ intensity\ per\ channel\ *\ factor}. The
channels with noise are removed from both the list with channels, as
well as the entire dataset.

\begin{verbatim}
filter_channels(<FREQUENCY_CHANNELS>, <TIME_SAMPLES>)
\end{verbatim}

\subsection{6.4 Filter individual
channels}\label{filter-individual-channels}

The \texttt{clipping.filter\_indv\_channels} function can be used to
replace all the remaining samples with noise.

The noise for each individual sample is identified by calculating the
mean intensity per channel. After that, samples are replaced with the
median of a channel if their intensity is higher than the
\texttt{average\ intensity\ per\ channel\ *\ factor}.

\begin{verbatim}
filter_indv_channels(<TIME_SAMPLES>)
\end{verbatim}

\section{7. Dedispersion}\label{dedispersion}

\subsection{7.1 Introduction}\label{introduction}

\begin{itemize}
\item
  Pulsars produce a narrow beam of electromagnetic radiation which
  rotates like a lighthouse beam, so a pulse is seen as it sweeps over a
  radiotelescope. The signal is spread over a wide frequency range.
\item
  If space was an empty vacuum, all the signals would travel at the same
  speed, but due to free electrons different frequencies travel at
  slightly different speeds (dispersion).
\end{itemize}

\subsection{7.2 Dedisperse}\label{dedisperse}

This method performs dedispersion on the filterbank data.

\subsubsection{7.2.1 Parameters}\label{parameters-3}

\begin{longtable}[]{@{}ll@{}}
\toprule
Parameters & Description\tabularnewline
\midrule
\endhead
Samples & Array or sequence containing the data to be
plotted.\tabularnewline
DM & Dispersion measure (cm-3 pc)\tabularnewline
\bottomrule
\end{longtable}

\subsubsection{7.2.2 Returns}\label{returns-1}

\begin{longtable}[]{@{}ll@{}}
\toprule
Variable & Description\tabularnewline
\midrule
\endhead
Samples & Dedispersed samples\tabularnewline
\bottomrule
\end{longtable}

\subsection{7.3 find\_dm}\label{find_dm}

This method finds the dispersion measure.

\subsubsection{7.3.1 Parameters}\label{parameters-4}

\begin{longtable}[]{@{}ll@{}}
\toprule
\begin{minipage}[b]{0.41\columnwidth}\raggedright\strut
Parameters\strut
\end{minipage} & \begin{minipage}[b]{0.41\columnwidth}\raggedright\strut
Description\strut
\end{minipage}\tabularnewline
\midrule
\endhead
\begin{minipage}[t]{0.41\columnwidth}\raggedright\strut
Samples\strut
\end{minipage} & \begin{minipage}[t]{0.41\columnwidth}\raggedright\strut
1-D array or sequence. Array or sequence containing the data to be
plotted.\strut
\end{minipage}\tabularnewline
\bottomrule
\end{longtable}

\subsubsection{7.3.2 Returns}\label{returns-2}

\begin{longtable}[]{@{}ll@{}}
\toprule
Variable & Description\tabularnewline
\midrule
\endhead
dm & Dispersion measure\tabularnewline
\bottomrule
\end{longtable}

\subsection{7.4 find\_line}\label{find_line}

This method finds a line starting from the sample index given in the
parameters. This will stop if there isn't a intensity within the
max\_delay higher than the average\_intensity.

\subsubsection{7.4.1 Parameters}\label{parameters-5}

\begin{longtable}[]{@{}ll@{}}
\toprule
\begin{minipage}[b]{0.41\columnwidth}\raggedright\strut
Parameters\strut
\end{minipage} & \begin{minipage}[b]{0.41\columnwidth}\raggedright\strut
Description\strut
\end{minipage}\tabularnewline
\midrule
\endhead
\begin{minipage}[t]{0.41\columnwidth}\raggedright\strut
Samples\strut
\end{minipage} & \begin{minipage}[t]{0.41\columnwidth}\raggedright\strut
1-D array or sequence. Array or sequence containing the data to be
plotted.\strut
\end{minipage}\tabularnewline
\begin{minipage}[t]{0.41\columnwidth}\raggedright\strut
start\_sample\_index\strut
\end{minipage} & \begin{minipage}[t]{0.41\columnwidth}\raggedright\strut
The start sample index\strut
\end{minipage}\tabularnewline
\begin{minipage}[t]{0.41\columnwidth}\raggedright\strut
max\_delay\strut
\end{minipage} & \begin{minipage}[t]{0.41\columnwidth}\raggedright\strut
The max delay of the recieved signal\strut
\end{minipage}\tabularnewline
\begin{minipage}[t]{0.41\columnwidth}\raggedright\strut
average\_intensity\strut
\end{minipage} & \begin{minipage}[t]{0.41\columnwidth}\raggedright\strut
The average intensity\strut
\end{minipage}\tabularnewline
\bottomrule
\end{longtable}

\subsubsection{7.4.2 Returns}\label{returns-3}

\begin{longtable}[]{@{}ll@{}}
\toprule
Variable & Description\tabularnewline
\midrule
\endhead
start\_sample\_index & The first frequency\tabularnewline
previous\_sample\_index & The higher frequency\tabularnewline
\bottomrule
\end{longtable}

\subsection{7.5
find\_estimation\_intensity}\label{find_estimation_intensity}

This method finds the average intensity for top x intensities. The
\texttt{average\_intensity} is considered a requirement for intensities
to be considered a pulsar.

\subsubsection{7.5.1 Parameters}\label{parameters-6}

\begin{longtable}[]{@{}ll@{}}
\toprule
Parameters & Description\tabularnewline
\midrule
\endhead
Samples & Array or sequence containing the data to be
plotted.\tabularnewline
\bottomrule
\end{longtable}

\subsubsection{7.5.2 Returns}\label{returns-4}

\begin{longtable}[]{@{}ll@{}}
\toprule
Variable & Description\tabularnewline
\midrule
\endhead
average\_intensity & The avarage intesity\tabularnewline
\bottomrule
\end{longtable}

\section{8. Pipeline}\label{pipeline}

\subsection{8.1 Introduction}\label{introduction-1}

The Pipeline module is used to execute the different modules in a
specific order. There are currently three different options for running
the pipeline.

These options include: * read multiple rows, \texttt{read\_n\_rows} *
read single rows, \texttt{read\_rows} * read all rows,
\texttt{read\_static}

The constructor of the pipeline module will recognize which method is
fit for running which method, by looking at the given arguments to the
constructor.

\begin{longtable}[]{@{}ll@{}}
\toprule
\begin{minipage}[b]{0.41\columnwidth}\raggedright\strut
Parameter\strut
\end{minipage} & \begin{minipage}[b]{0.41\columnwidth}\raggedright\strut
Description\strut
\end{minipage}\tabularnewline
\midrule
\endhead
\begin{minipage}[t]{0.41\columnwidth}\raggedright\strut
filename\strut
\end{minipage} & \begin{minipage}[t]{0.41\columnwidth}\raggedright\strut
The path to the filterbank file.\strut
\end{minipage}\tabularnewline
\begin{minipage}[t]{0.41\columnwidth}\raggedright\strut
as\_stream\strut
\end{minipage} & \begin{minipage}[t]{0.41\columnwidth}\raggedright\strut
This parameter decides whether the filterbank should be read as
stream.\strut
\end{minipage}\tabularnewline
\begin{minipage}[t]{0.41\columnwidth}\raggedright\strut
DM\strut
\end{minipage} & \begin{minipage}[t]{0.41\columnwidth}\raggedright\strut
The dispersion measure (DM) is used for performing dedispersion.\strut
\end{minipage}\tabularnewline
\begin{minipage}[t]{0.41\columnwidth}\raggedright\strut
scale\strut
\end{minipage} & \begin{minipage}[t]{0.41\columnwidth}\raggedright\strut
The scale is used for performing downsampling the time series.\strut
\end{minipage}\tabularnewline
\begin{minipage}[t]{0.41\columnwidth}\raggedright\strut
n\strut
\end{minipage} & \begin{minipage}[t]{0.41\columnwidth}\raggedright\strut
The \texttt{n} is the rowsize of chunks for reading the filterbank as
stream.\strut
\end{minipage}\tabularnewline
\begin{minipage}[t]{0.41\columnwidth}\raggedright\strut
size\strut
\end{minipage} & \begin{minipage}[t]{0.41\columnwidth}\raggedright\strut
The size parameter is used for deciding the size of the
filterbank.\strut
\end{minipage}\tabularnewline
\bottomrule
\end{longtable}

After deciding which method to run for running the filterbank in a
pipeline, it will measure the time it takes to run each method using
\texttt{measure\_method}. After running all the different methods, the
constructor will append the results (a dictionary) to a txt file.

\subsection{8.2 Read rows}\label{read-rows}

The \texttt{read\_rows} method reads the Filterbank data row per row.
Because it only reads the filterbank per row, it is unable to execute
most methods. The alternative for this method is the
\texttt{read\_n\_rows} method, which is able to run all methods.

\begin{verbatim}
pipeline.Pipeline(<filterbank_file>, as_stream=True)
\end{verbatim}

\subsection{8.3 Read n rows}\label{read-n-rows}

The \texttt{read\_n\_rows} method first splits all the filterbank data
into chunks of n samples. After splitting the filterbank data in chunks,
it will run the different modules of the pipeline for each chunk. The
remaining data, that which does not fit into the sample size, is
currently ignored.

The \texttt{n} or sample size should be a power of 2 multiplied with the
given scale for the downsampling.

\begin{verbatim}
pipeline.Pipeline(<filterbank_file>, n=<size> , as_stream=True)
\end{verbatim}

\subsection{8.4 Read static}\label{read-static}

The \texttt{read\_static} method reads the entire filterbank at once,
and applies each method to the entire dataset. If the filterbank file is
too large for running it in-memory, the alternative is using
\texttt{read\_n\_rows}.

\begin{verbatim}
pipeline.Pipeline(<filterbank_file>)
\end{verbatim}

\subsection{8.5 Measure methods}\label{measure-methods}

The \texttt{measure\_methods} is ran for each of the above methods, and
calculates the time it takes to run each of the different methods. For
each method it will create a key using the name of the method, and save
the time it took to run the method as a value. At the end, it will
returns a dictionary with all the keys and values.

\subsection{7.6 Overview of pipeline}\label{overview-of-pipeline}

Apart from the different modules described in the previous paragraphs,
additional modules are required for this library to make detecting
pulsar signals possible. However, these additional modules have not been
developed yet, and are required to be developed in the future. In this
paragraph the additional modules are listed and described.

Modules that are missing in the pipeline are highlighted using a
\texttt{*}.

\begin{verbatim}
1. Read Filterbank as stream
2. Reduce RFI using clipping
3. Dedisperse radio signal
4. Transform dedispersed signal to TimeSeries
5. Run fast Fourier transformation on TimeSeries
6. * Identify and save birdies in file
7. * Perform Harmonic Summing
8. * Search and identify single and periodic signals
9. * Phase-fold remaining signals
10.* Do Transient searches
\end{verbatim}

\section{9. Generating mock data}\label{generating-mock-data}

\subsection{9.1 Creating a filterbank
file}\label{creating-a-filterbank-file}

To generate a filterbank file you may use the following example code:

\begin{Shaded}
\begin{Highlighting}[]
\ImportTok{import}\NormalTok{ filterbank.header }\ImportTok{as}\NormalTok{ header}

\NormalTok{header }\OperatorTok{=}\NormalTok{ \{}
\NormalTok{    b}\StringTok{'source_name'}\NormalTok{: b}\StringTok{'P: 80.0000 ms, DM: 200.000'}\NormalTok{,}
\NormalTok{    b}\StringTok{'machine_id'}\NormalTok{: }\DecValTok{10}\NormalTok{,}
\NormalTok{    b}\StringTok{'telescope_id'}\NormalTok{: }\DecValTok{4}\NormalTok{,}
\NormalTok{    b}\StringTok{'data_type'}\NormalTok{: }\DecValTok{1}\NormalTok{,}
\NormalTok{    b}\StringTok{'fch1'}\NormalTok{: }\DecValTok{400}\NormalTok{,}
\NormalTok{    b}\StringTok{'foff'}\NormalTok{: }\OperatorTok{-}\FloatTok{0.062}\NormalTok{,}
\NormalTok{    b}\StringTok{'nchans'}\NormalTok{: }\DecValTok{128}\NormalTok{,}
\NormalTok{    b}\StringTok{'tstart'}\NormalTok{: }\FloatTok{6000.0}\NormalTok{,}
\NormalTok{    b}\StringTok{'tsamp'}\NormalTok{: }\FloatTok{8e-05}\NormalTok{,}
\NormalTok{    b}\StringTok{'nifs'}\NormalTok{: }\DecValTok{1}\NormalTok{,}
\NormalTok{    b}\StringTok{'nbits'}\NormalTok{: }\DecValTok{8}
\NormalTok{\}}

\NormalTok{header.generate_file(}\StringTok{"file_path/file_name"}\NormalTok{, header)}
\end{Highlighting}
\end{Shaded}

The header data is passed in a header dict. For a more detailed
explanation on the file header, please consult
\href{docs/filterbank.md\#22-read-the-header-from-filterbank-data}{chapter
2.2}.

\subsection{9.2 Generate signal}\label{generate-signal}

The generate\_signal method generates a mock signal based on the
following parameters:

\begin{longtable}[]{@{}ll@{}}
\toprule
Variable & Description\tabularnewline
\midrule
\endhead
noise\_level & the max amplitude of the generated noise\tabularnewline
period & period of the signal\tabularnewline
t\_obs & observation time in s\tabularnewline
n\_pts & intervals between samples\tabularnewline
\bottomrule
\end{longtable}

\subsection{9.3 Generate header}\label{generate-header}

The generate\_header method generates a header string based on the
header dict provided in the example in chapter 8.1. The dict provides
keys encoded in bytes and the method converts each keyword to a string
using the keyword\_to\_string method (see below).

\subsection{9.4 Keyword to string}\label{keyword-to-string}

The keyword\_to\_string method converts a keyword from the header dict
to a serialized string.

\subsection{9.5 Write data}\label{write-data}

Once all the required data is generated to a filterbank file. The data
is written to the file as bytes.

\end{document}
